\documentclass[handout, xcolor=pdftex,dvipsnames,table]{beamer}  %presentation,
%\documentclass[handout, xcolor=pdftex,dvipsnames,table]  %presentation,
\usetheme{Hannover}

\usepackage[utf8]{inputenc}
\usepackage[T1]{fontenc}
\usepackage[english, norsk]{babel}
\usepackage{xspace}
\usepackage{booktabs}
\usepackage{rotating}
\usepackage{graphicx}






\begin{document}

\title[Rygg\\Tromsø \\\today]{\textit{NKR: Degenerativ rygg} \\
Månedsrapport, februar 2021 \\
Tromsø }
%{\today}\\[2cm] % Date, change the \today to a set date if you want to be precise \date{}
\maketitle


\begin{tiny}

\begin{frame}[fragile] {Ansvar og bruk av data}

Denne rapporten er utarbeidet på oppdrag fra Norsk kvalitetsregister for ryggkirurgi (NKR) og genereres automatisk
fra Rapporteket.

Fagrådet for NKR er ansvarlig for alle vurderinger og tolkninger av data, og også feil i resultat som skyldes feil i datagrunnlaget. Den viktigste feilkilden i rapporten vil trolig være ufullstendige datasett for mange av de registrerende enheter – det vil si at ikke alle data er ferdigstilte for aktuell månad. Dette etterslepet vil variere markert både i tid og volum, og gjør at data blir mer eller mindre representative for de ulike enhetene.
Rapporten har følgelig usikkert datagrunnlag og er ment til internt bruk i enhetene.
Rapporten inneholder anonymiserte samledata. I tilfeller der utvalget inneholder få
registreringer kombinert med eksempelvis demografiske data, kan det ikke utelukkes at rapporterte data kan tilbakeidentifiseres til enkeltpersoner.

\textit{\textbf{Av ovenfor nevnte årsaker, ber NKR om at denne rapporten ikke blir brukt
offentlig i noen sammenheng eller på noe vis blir offentliggjort.}}

\end{frame}


\section{Datagrunnlag og innhold}

\begin{frame}[fragile] {Innhold}
Dette er en sammenstilling av resultater  fra Norsk Kvalitetsregister for Ryggkirurgi, Degenerativ Rygg.
Alle resultater er baserte på data fra registeret og er gyldige per rapportdato for
opphold som er ferdigstilte t.o.m. dagen før rapportdato i QReg. Data er hentet rett fra registeret og er ikke kvalitetssikret.

Datoer/årstall er basert på operasjonsdato. Resultatene som vises er i all hovedsak for de siste 12 måneder.
Tidsutvalg for rapportene er spesifisert øverst i hver enkelt figur.

Rapporten viser følgende:
\begin{itemize}
\item Registreringsoversikt
\item Registreingsforsinkelse (egen sammenlignet med resten)
\item Nøkkeltall, kvartalsvis
\item Ventetid fra henvisning til time på poliklinikk, kvartalsvis
\item Ventetid fra operasjon ble bestemt til utført, kvartalsvis
\item Andel Ø-hjelp, (egen sammenlignet med resten), kvartalsvis
\item Inngrepstype
\item Postoperativ liggetid prolaps (egen sammenlignet med resten), kvartalsvis
\item Postoperativ liggetid spinal stenose (egen sammenlignet med resten), kvartalsvis
\item Symptomvarighet
\item Durarift prolaps (egen sammenlignet med resten), kvartalsvis
\item Durarift spinal stenose (egen sammenlignet med resten), kvartalsvis
\item Pasientrapportert sårinfeksjon 3 mnd, dyp og overflatisk, kvartalsvis
\item Gjennomsnittlig forbedring av ODI prolaps (egen sammenlignet med resten), kvartalsvis
\item Gjennomsnittlig forbedring av ODI spinal stenose (egen sammenlignet med resten), kvartalsvis
\item Tilfredshet hos pasienten 3 mnd etter opr, kvartalsvis
\item Nytte av opr. 3 mnd etter, kvartalsvis
\end{itemize}

Dette er bare et lite utvalg resultater. På Rapporteket kan du finne mer spesifikke resultater for disse og mange andre variable.

\end{frame}


\section{Registreringsoversikter}

\begin{frame}[fragile] {Registreringsoversikt og nøkkeltall}
For Tromsø er siste ferdigstilte operasjonsdato i denne rapporten
23.02.2021.

% latex table generated in R 4.0.4 by xtable 1.8-4 package
% Sat Mar 27 10:10:31 2021
\begin{table}[ht]
\centering
\begin{tabular}{lrrrrrr}
  \hline
 & sep. 20 & okt. 20 & nov. 20 & des. 20 & jan. 21 & feb. 21 \\ 
  \hline
Tromsø & 53 & 36 & 37 & 28 & 35 & 25 \\ 
   \hline
\end{tabular}
\caption{Antall registreringer, Tromsø.} 
\label{tab:RegEget}
\end{table}
% latex table generated in R 4.0.4 by xtable 1.8-4 package
% Sat Mar 27 10:10:31 2021
\begin{table}[ht]
\centering
\begin{tabular}{lrrrrr}
  \hline
 & 20-1 & 20-2 & 20-3 & 20-4 & 21-1 \\ 
  \hline
Antall operasjoner & 59 & 59 & 89 & 101 & 60 \\ 
  Alder $>$ 70 år & 22 & 25 & 22 & 16 & 35 \\ 
  Alder (gj.sn) & 55 & 57 & 58 & 53 & 55 \\ 
  Kvinneandel (\%) & 46 & 53 & 40 & 48 & 53 \\ 
  Liggedøgn, totalt & 180 & 181 & 177 & 126 & 110 \\ 
  Liggetid, postop., (gj.sn.) & 2 & 2 & 1 & 1 & 1 \\ 
  Reg.forsinkelse (gj.sn., dager) & 54 & 59 & 23 & 21 & 8 \\ 
   \hline
\end{tabular}
\caption{Nøkkeltall, Tromsø.} 
\label{tab:NokkeltallEget}
\end{table}

\end{frame}


@
\end{frame}



\begin{frame}[fragile] {Registreringsforsinkelse}

\begin{figure}[ht]
\centering
\includegraphics[scale=0.35]{RegForsinkelse.pdf}
\caption{Forsinket registrering av operasjoner, egen avdeling og resten av landet.}
\end{figure}

\end{frame}




\begin{frame}[fragile] {Ventetid fra henvisning til time på poliklinikk}
\begin{figure}[ht]
\centering
\includegraphics[scale=0.35]{VentetidHenvTimePol.pdf}
\caption{Ventetid fra henvisning til time på poliklinikk, egen avd. og resten av landet.}
\end{figure}
\end{frame}

\begin{frame}[fragile] {Ventetid fra operasjon ble bestemt til utført}
\begin{figure}[ht]
\centering
\includegraphics[scale=0.35]{VentetidSpesOp.pdf}
\caption{Ventetid fra operasjon ble bestemt til utført, egen avd. og resten av landet.}
\end{figure}
\end{frame}

\begin{frame}[fragile] {Andel ø-hjelp}
\begin{figure}[ht]
\centering
\includegraphics[scale=0.35]{Ohjelp.pdf}
\caption{Andel øyeblikkelig hjelp, egen avd. og resten av landet.}
\end{figure}
\end{frame}

\begin{frame}[fragile] {Inngrepstyper}
\begin{figure}[ht]
\centering
\includegraphics[scale=0.35]{Inngrepstyper.pdf}
\caption{Inngrepstyper, egen avd. og resten av landet.}
\end{figure}
\end{frame}



\begin{frame}[fragile] {Liggetid på avdelinga}

Liggetida er tida fra pasienten legges inn på avdelinga til pasienten skrives ut.
Liggetida inkluderer derfor både preoperativ og postoperativ liggetid).

% latex table generated in R 4.0.4 by xtable 1.8-4 package
% Sat Mar 27 10:10:33 2021
\begin{table}[ht]
\centering
\begin{tabular}{lrrr}
  \hline
 & Gj.snitt & Median & Totalsum \\ 
  \hline
Egen, siste måned & 2.3 & 2.0 & 57 \\ 
  Egen & 1.6 & 1.0 & 349 \\ 
  Resten av landet & 2.0 & 1.0 & 5403 \\ 
   \hline
\end{tabular}
\caption{Liggedøgn, siste 6 måneder.} 
\label{tab:liggetidKj}
\end{table}


\end{frame}

\begin{frame}[fragile] {Postoperativ liggetid, prolaps}
\begin{figure}[ht]
\centering
\includegraphics[scale=0.35]{LiggetidPostOp_pro.pdf}
\caption{Postoperativ liggetid, prolaps, egen avd. og resten av landet.}
\end{figure}
\end{frame}

\begin{frame}[fragile] {Postoperativ liggetid, spinal stenose}
\begin{figure}[ht]
\centering
\includegraphics[scale=0.35]{LiggetidPostOp_SS.pdf}
\caption{Postoperativ liggetid, spinal stenose, egen avd. og resten av landet.}
\end{figure}
\end{frame}


\begin{frame}[fragile] {Symptomvarighet}
\begin{figure}[ht]
\centering
\includegraphics[scale=0.35]{SymptVarighUtstrTilBen.pdf}
\caption{Varighet av utstrålende smerter til ben, egen avdeling og resten av landet.}
\end{figure}
\end{frame}

\begin{frame}[fragile] {Durarift prolaps}
\begin{figure}[ht]
\centering
\includegraphics[scale=0.35]{peropKompDura_pro.pdf}
\caption{Peroperativ komplikasjon, durarift, prolapsopererte, egen avd og resten av landet.}
\end{figure}
\end{frame}

\begin{frame}[fragile] {Durarift spinal stenose}
\begin{figure}[ht]
\centering
\includegraphics[scale=0.35]{peropKompDura_SS.pdf}
\caption{Peroperativ komplikasjon, durarift, spinal stenose, egen avd og resten av landet.}
\end{figure}
\end{frame}



